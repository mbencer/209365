\documentclass[a4paper]{book}
\usepackage{makeidx}
\usepackage{graphicx}
\usepackage{multicol}
\usepackage{float}
\usepackage{listings}
\usepackage{color}
\usepackage{textcomp}
\usepackage{alltt}
\usepackage{times}
\usepackage{ifpdf}
\ifpdf
\usepackage[pdftex,
            pagebackref=true,
            colorlinks=true,
            linkcolor=blue,
            unicode
           ]{hyperref}
\else
\usepackage[ps2pdf,
            pagebackref=true,
            colorlinks=true,
            linkcolor=blue,
            unicode
           ]{hyperref}
\usepackage{pspicture}
\fi
\usepackage[utf8]{inputenc}
\usepackage{polski}
\usepackage[T1]{fontenc}

\usepackage{doxygen}
\lstset{language=C++,inputencoding=utf8,basicstyle=\footnotesize,breaklines=true,breakatwhitespace=true,tabsize=4,numbers=left }
\makeindex
\setcounter{tocdepth}{3}
\renewcommand{\footrulewidth}{0.4pt}
\begin{document}
\hypersetup{pageanchor=false}
\begin{titlepage}
\vspace*{7cm}
\begin{center}
{\Large Benchmark \\[1ex]\large 1 }\\
\vspace*{1cm}
{\large Wygenerowano przez Doxygen 1.7.1}\\
\vspace*{0.5cm}
{\small Wed Mar 18 2015 15:32:36}\\
\end{center}
\end{titlepage}
\clearemptydoublepage
\pagenumbering{roman}
\tableofcontents
\clearemptydoublepage
\pagenumbering{arabic}
\hypersetup{pageanchor=true}
\chapter{Program liczacy czas przebiegu funkcji dla roznych danych, w celu zobaczenia zlozonosci obliczeniowej.}
\label{index}\hypertarget{index}{}\begin{DoxyAuthor}{Autor}
Lukasz Sak 
\end{DoxyAuthor}
\begin{DoxyVersion}{Wersja}
1
\end{DoxyVersion}
Program posiada definicje struktur danych: \hyperlink{class_stos}{Stos}, \hyperlink{class_lista}{Lista}, \hyperlink{class_kolejka}{Kolejka}. Struktury posiadaja wiekszosc takich samych metod. PUSH() -\/ wrzucajaca dana do struktury, POP() -\/ usuwajacy odpowiednia dana ze struktury, SIZE() -\/ zwracajacy ilosc elementow w strukturze, SHOW() -\/ wyswietlajacy elementy struktury. Struktury sa zrobione na szablonach, zatem mozna uzywac kilku typow danych. 
\chapter{Indeks klas}
\section{Lista klas}
Tutaj znajdują się klasy, struktury, unie i interfejsy wraz z ich krótkimi opisami\-:\begin{DoxyCompactList}
\item\contentsline{section}{\hyperlink{class_benchmark}{Benchmark$<$ T $>$} \\*Klasa do przeprowadzenia testów na kontenerach danych }{\pageref{class_benchmark}}{}
\item\contentsline{section}{\hyperlink{class_kolejka}{Kolejka$<$ T\-Y\-P $>$} \\*Klasa \hyperlink{class_kolejka}{Kolejka} }{\pageref{class_kolejka}}{}
\item\contentsline{section}{\hyperlink{class_lista}{Lista$<$ T\-Y\-P $>$} \\*Klasa \hyperlink{class_lista}{Lista} }{\pageref{class_lista}}{}
\item\contentsline{section}{\hyperlink{class_stos}{Stos$<$ T\-Y\-P $>$} \\*Klasa \hyperlink{class_stos}{Stos} }{\pageref{class_stos}}{}
\item\contentsline{section}{\hyperlink{class_timer}{Timer} \\*Klasa do pomiaru różnicy czasów }{\pageref{class_timer}}{}
\end{DoxyCompactList}

\chapter{Indeks plików}
\section{Lista plików}
Tutaj znajduje się lista wszystkich plików z ich krótkimi opisami:\begin{DoxyCompactList}
\item\contentsline{section}{\hyperlink{_benchmark_8cpp}{Benchmark.cpp} (Metody klasy \hyperlink{class_benchmark}{Benchmark} )}{\pageref{_benchmark_8cpp}}{}
\item\contentsline{section}{\hyperlink{_benchmark_8hh}{Benchmark.hh} (Definicja klasy \hyperlink{class_benchmark}{Benchmark} )}{\pageref{_benchmark_8hh}}{}
\item\contentsline{section}{\hyperlink{_kolejka_8hh}{Kolejka.hh} (Definicja klasy \hyperlink{class_kolejka}{Kolejka} )}{\pageref{_kolejka_8hh}}{}
\item\contentsline{section}{\hyperlink{_lista_8cpp}{Lista.cpp} }{\pageref{_lista_8cpp}}{}
\item\contentsline{section}{\hyperlink{_lista_8hh}{Lista.hh} (Definicja klasy \hyperlink{class_lista}{Lista} )}{\pageref{_lista_8hh}}{}
\item\contentsline{section}{\hyperlink{_stos_8hh}{Stos.hh} (Definicja klasy \hyperlink{class_stos}{Stos} )}{\pageref{_stos_8hh}}{}
\item\contentsline{section}{\hyperlink{_struktury_8cpp}{Struktury.cpp} }{\pageref{_struktury_8cpp}}{}
\item\contentsline{section}{\hyperlink{_test_8cpp}{Test.cpp} }{\pageref{_test_8cpp}}{}
\end{DoxyCompactList}

\chapter{Dokumentacja klas}
\hypertarget{class_benchmark}{\section{Dokumentacja szablonu klasy Benchmark$<$ T $>$}
\label{class_benchmark}\index{Benchmark$<$ T $>$@{Benchmark$<$ T $>$}}
}


Klasa do przeprowadzenia testów na kontenerach danych.  




{\ttfamily \#include $<$Benchmark.\-hh$>$}

\subsection*{Metody publiczne}
\begin{DoxyCompactItemize}
\item 
\hyperlink{class_benchmark_a15ccd21b999908da42c58bd5d127c674}{Benchmark} (unsigned int $\ast$test\-\_\-list\-\_\-sizes, unsigned int size, unsigned int number\-\_\-of\-\_\-memeasurements)
\begin{DoxyCompactList}\small\item\em Konstruktor klasy \hyperlink{class_benchmark}{Benchmark}. \end{DoxyCompactList}\item 
void \hyperlink{class_benchmark_ac6a294f9ac4fbba06f0a608eb241d3b0}{test\-Quick\-Sort} (\hyperlink{class_lista}{Lista}$<$ T $>$ $\ast$list, P\-I\-V\-O\-T piv)
\begin{DoxyCompactList}\small\item\em metoda do testowania algorytmu Quick Sort \end{DoxyCompactList}\item 
void \hyperlink{class_benchmark_a90eb1776343e90f4fcbf3fb95aa52bef}{test\-Merge\-Sort} (\hyperlink{class_lista}{Lista}$<$ T $>$ $\ast$list)
\begin{DoxyCompactList}\small\item\em metoda do testowania algorytmyu Merge Sort \end{DoxyCompactList}\end{DoxyCompactItemize}


\subsection{Opis szczegółowy}
\subsubsection*{template$<$class T$>$class Benchmark$<$ T $>$}

Klasa do przeprowadzenia testów na kontenerach danych. 

Klasa odpwiada za pomiar czasu koniecznego do umieszczenia określonej liczby danych na stosie, kolejce oraz liście 

\subsection{Dokumentacja konstruktora i destruktora}
\hypertarget{class_benchmark_a15ccd21b999908da42c58bd5d127c674}{\index{Benchmark@{Benchmark}!Benchmark@{Benchmark}}
\index{Benchmark@{Benchmark}!Benchmark@{Benchmark}}
\subsubsection[{Benchmark}]{\setlength{\rightskip}{0pt plus 5cm}template$<$typename T $>$ {\bf Benchmark}$<$ T $>$\-::{\bf Benchmark} (
\begin{DoxyParamCaption}
\item[{unsigned int $\ast$}]{test\-\_\-list\-\_\-sizes, }
\item[{unsigned int}]{size, }
\item[{unsigned int}]{number\-\_\-of\-\_\-memeasurements}
\end{DoxyParamCaption}
)}}\label{class_benchmark_a15ccd21b999908da42c58bd5d127c674}


Konstruktor klasy \hyperlink{class_benchmark}{Benchmark}. 


\begin{DoxyParams}{Parametry}
{\em test\-\_\-list\-\_\-sizes} & tablica przechowująca rozmiary list do testów \\
\hline
{\em size} & liczba list równych rozmiarów do przetestowania \\
\hline
{\em number\-\_\-of\-\_\-memeasurements} & liczba pomiarów branych do średniej \\
\hline
\end{DoxyParams}


\subsection{Dokumentacja funkcji składowych}
\hypertarget{class_benchmark_a90eb1776343e90f4fcbf3fb95aa52bef}{\index{Benchmark@{Benchmark}!test\-Merge\-Sort@{test\-Merge\-Sort}}
\index{test\-Merge\-Sort@{test\-Merge\-Sort}!Benchmark@{Benchmark}}
\subsubsection[{test\-Merge\-Sort}]{\setlength{\rightskip}{0pt plus 5cm}template$<$typename T $>$ void {\bf Benchmark}$<$ T $>$\-::test\-Merge\-Sort (
\begin{DoxyParamCaption}
\item[{{\bf Lista}$<$ T $>$ $\ast$}]{list}
\end{DoxyParamCaption}
)}}\label{class_benchmark_a90eb1776343e90f4fcbf3fb95aa52bef}


metoda do testowania algorytmyu Merge Sort 


\begin{DoxyParams}{Parametry}
{\em list} & lista na której testujemy sortowanie \\
\hline
\end{DoxyParams}
\hypertarget{class_benchmark_ac6a294f9ac4fbba06f0a608eb241d3b0}{\index{Benchmark@{Benchmark}!test\-Quick\-Sort@{test\-Quick\-Sort}}
\index{test\-Quick\-Sort@{test\-Quick\-Sort}!Benchmark@{Benchmark}}
\subsubsection[{test\-Quick\-Sort}]{\setlength{\rightskip}{0pt plus 5cm}template$<$typename T $>$ void {\bf Benchmark}$<$ T $>$\-::test\-Quick\-Sort (
\begin{DoxyParamCaption}
\item[{{\bf Lista}$<$ T $>$ $\ast$}]{list, }
\item[{P\-I\-V\-O\-T}]{piv}
\end{DoxyParamCaption}
)}}\label{class_benchmark_ac6a294f9ac4fbba06f0a608eb241d3b0}


metoda do testowania algorytmu Quick Sort 


\begin{DoxyParams}{Parametry}
{\em list} & lista na której testujemy sortowanie \\
\hline
{\em piv} & jaką motedę wyboru elementu osiowego wybieramy \\
\hline
\end{DoxyParams}


Dokumentacja dla tej klasy została wygenerowana z pliku\-:\begin{DoxyCompactItemize}
\item 
/home/mateusz/git/209365/\-Lab3/prj/inc/\hyperlink{_benchmark_8hh}{Benchmark.\-hh}\end{DoxyCompactItemize}

\chapter{Dokumentacja plików}
\hypertarget{_benchmark_8cpp}{
\section{Dokumentacja pliku Benchmark.cpp}
\label{_benchmark_8cpp}\index{Benchmark.cpp@{Benchmark.cpp}}
}


Metody klasy \hyperlink{class_benchmark}{Benchmark}.  


{\ttfamily \#include \char`\"{}Benchmark.hh\char`\"{}}\par
{\ttfamily \#include $<$cstdlib$>$}\par
{\ttfamily \#include $<$fstream$>$}\par
{\ttfamily \#include $<$ctime$>$}\par
Wykres zależności załączania dla Benchmark.cpp:
\nopagebreak
\begin{figure}[H]
\begin{center}
\leavevmode
\includegraphics[width=348pt]{_benchmark_8cpp__incl}
\end{center}
\end{figure}


\subsection{Opis szczegółowy}
Plik zawiera definicje metod klasy \hyperlink{class_benchmark}{Benchmark} 

Definicja w pliku \hyperlink{_benchmark_8cpp_source}{Benchmark.cpp}.


\hypertarget{_benchmark_8hh}{\section{Dokumentacja pliku /home/mateusz/git/209365/\-Lab3/prj/inc/\-Benchmark.hh}
\label{_benchmark_8hh}\index{/home/mateusz/git/209365/\-Lab3/prj/inc/\-Benchmark.\-hh@{/home/mateusz/git/209365/\-Lab3/prj/inc/\-Benchmark.\-hh}}
}


Deklaracja i definicja (razem, bo szablon) klasy \hyperlink{class_benchmark}{Benchmark}.  


{\ttfamily \#include \char`\"{}../inc/\-Lista.\-hh\char`\"{}}\\*
{\ttfamily \#include \char`\"{}../inc/quick\-\_\-sort.\-hh\char`\"{}}\\*
{\ttfamily \#include \char`\"{}../inc/merge\-\_\-sort.\-hh\char`\"{}}\\*
{\ttfamily \#include \char`\"{}../inc/\-Timer.\-hh\char`\"{}}\\*
{\ttfamily \#include $<$iostream$>$}\\*
{\ttfamily \#include $<$cstdlib$>$}\\*
{\ttfamily \#include $<$cstdio$>$}\\*
{\ttfamily \#include $<$ctime$>$}\\*
{\ttfamily \#include $<$string$>$}\\*
\subsection*{Komponenty}
\begin{DoxyCompactItemize}
\item 
class \hyperlink{class_benchmark}{Benchmark$<$ T $>$}
\begin{DoxyCompactList}\small\item\em Klasa do przeprowadzenia testów na kontenerach danych. \end{DoxyCompactList}\end{DoxyCompactItemize}


\subsection{Opis szczegółowy}
Deklaracja i definicja (razem, bo szablon) klasy \hyperlink{class_benchmark}{Benchmark}. \hyperlink{_benchmark_8hh}{Benchmark.\-hh} 
\hypertarget{main_8cpp}{
\section{Dokumentacja pliku main.cpp}
\label{main_8cpp}\index{main.cpp@{main.cpp}}
}
{\ttfamily \#include $<$iostream$>$}\par
{\ttfamily \#include \char`\"{}Benchmark.hh\char`\"{}}\par
Wykres zależności załączania dla main.cpp:
\nopagebreak
\begin{figure}[H]
\begin{center}
\leavevmode
\includegraphics[width=182pt]{main_8cpp__incl}
\end{center}
\end{figure}
\subsection*{Definicje}
\begin{DoxyCompactItemize}
\item 
\#define \hyperlink{main_8cpp_a77b984cfb88c2ca29fea115a7019ba8f}{STALA}~5
\end{DoxyCompactItemize}
\subsection*{Funkcje}
\begin{DoxyCompactItemize}
\item 
double \hyperlink{main_8cpp_ae36bd4cd4ba4a0663e79b579bf31fc20}{funkcja} (double x)
\item 
int \hyperlink{main_8cpp_ae66f6b31b5ad750f1fe042a706a4e3d4}{main} ()
\end{DoxyCompactItemize}


\subsection{Dokumentacja definicji}
\hypertarget{main_8cpp_a77b984cfb88c2ca29fea115a7019ba8f}{
\index{main.cpp@{main.cpp}!STALA@{STALA}}
\index{STALA@{STALA}!main.cpp@{main.cpp}}
\subsubsection[{STALA}]{\setlength{\rightskip}{0pt plus 5cm}\#define STALA~5}}
\label{main_8cpp_a77b984cfb88c2ca29fea115a7019ba8f}


Definicja w linii 4 pliku main.cpp.



\subsection{Dokumentacja funkcji}
\hypertarget{main_8cpp_ae36bd4cd4ba4a0663e79b579bf31fc20}{
\index{main.cpp@{main.cpp}!funkcja@{funkcja}}
\index{funkcja@{funkcja}!main.cpp@{main.cpp}}
\subsubsection[{funkcja}]{\setlength{\rightskip}{0pt plus 5cm}double funkcja (
\begin{DoxyParamCaption}
\item[{double}]{ x}
\end{DoxyParamCaption}
)}}
\label{main_8cpp_ae36bd4cd4ba4a0663e79b579bf31fc20}


Definicja w linii 7 pliku main.cpp.



Oto graf wywoływań tej funkcji:
\nopagebreak
\begin{figure}[H]
\begin{center}
\leavevmode
\includegraphics[width=202pt]{main_8cpp_ae36bd4cd4ba4a0663e79b579bf31fc20_icgraph}
\end{center}
\end{figure}


\hypertarget{main_8cpp_ae66f6b31b5ad750f1fe042a706a4e3d4}{
\index{main.cpp@{main.cpp}!main@{main}}
\index{main@{main}!main.cpp@{main.cpp}}
\subsubsection[{main}]{\setlength{\rightskip}{0pt plus 5cm}int main (
\begin{DoxyParamCaption}
{}
\end{DoxyParamCaption}
)}}
\label{main_8cpp_ae66f6b31b5ad750f1fe042a706a4e3d4}


Definicja w linii 11 pliku main.cpp.



Oto graf wywołań dla tej funkcji:
\nopagebreak
\begin{figure}[H]
\begin{center}
\leavevmode
\includegraphics[width=292pt]{main_8cpp_ae66f6b31b5ad750f1fe042a706a4e3d4_cgraph}
\end{center}
\end{figure}



\input{strona_8dox}
\printindex
\end{document}
