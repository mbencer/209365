\documentclass[a4paper]{book}
\usepackage{makeidx}
\usepackage{graphicx}
\usepackage{multicol}
\usepackage{float}
\usepackage{listings}
\usepackage{color}
\usepackage{textcomp}
\usepackage{alltt}
\usepackage{times}
\usepackage{ifpdf}
\ifpdf
\usepackage[pdftex,
            pagebackref=true,
            colorlinks=true,
            linkcolor=blue,
            unicode
           ]{hyperref}
\else
\usepackage[ps2pdf,
            pagebackref=true,
            colorlinks=true,
            linkcolor=blue,
            unicode
           ]{hyperref}
\usepackage{pspicture}
\fi
\usepackage[utf8]{inputenc}
\usepackage{polski}
\usepackage[T1]{fontenc}

\usepackage{doxygen}
\lstset{language=C++,inputencoding=utf8,basicstyle=\footnotesize,breaklines=true,breakatwhitespace=true,tabsize=4,numbers=left }
\makeindex
\setcounter{tocdepth}{3}
\renewcommand{\footrulewidth}{0.4pt}
\begin{document}
\hypersetup{pageanchor=false}
\begin{titlepage}
\vspace*{7cm}
\begin{center}
{\Large Struktury \\[1ex]\large 1 }\\
\vspace*{1cm}
{\large Wygenerowano przez Doxygen 1.7.1}\\
\vspace*{0.5cm}
{\small Thu Mar 26 2015 06:26:58}\\
\end{center}
\end{titlepage}
\clearemptydoublepage
\pagenumbering{roman}
\tableofcontents
\clearemptydoublepage
\pagenumbering{arabic}
\hypersetup{pageanchor=true}
\chapter{Program tworzacy struktury danych.}
\label{index}\hypertarget{index}{}\begin{DoxyAuthor}{Autor}
Lukasz Sak 
\end{DoxyAuthor}
\begin{DoxyVersion}{Wersja}
1
\end{DoxyVersion}
Program posiada definicje struktur danych: \hyperlink{class_stos}{Stos}, \hyperlink{class_lista}{Lista}, \hyperlink{class_kolejka}{Kolejka}. Struktury posiadaja wiekszosc takich samych metod. PUSH() -\/ wrzucajaca dana do struktury, POP() -\/ usuwajacy odpowiednia dana ze struktury, SIZE() -\/ zwracajacy ilosc elementow w strukturze, SHOW() -\/ wyswietlajacy elementy struktury. Struktury sa zrobione na szablonach, zatem mozna uzywac kilku typow danych. 
\chapter{Indeks klas}
\section{Hierarchia klas}
Ta lista dziedziczenia posortowana jest z grubsza, choć nie całkowicie, alfabetycznie:\begin{DoxyCompactList}
\item \contentsline{section}{Benchmark}{\pageref{class_benchmark}}{}
\item \contentsline{section}{Lista$<$ TYP $>$}{\pageref{class_lista}}{}
\begin{DoxyCompactList}
\item \contentsline{section}{Kolejka$<$ TYP $>$}{\pageref{class_kolejka}}{}
\item \contentsline{section}{Stos$<$ TYP $>$}{\pageref{class_stos}}{}
\end{DoxyCompactList}
\end{DoxyCompactList}

\chapter{Indeks klas}
\section{Lista klas}
Tutaj znajdują się klasy, struktury, unie i interfejsy wraz z ich krótkimi opisami\-:\begin{DoxyCompactList}
\item\contentsline{section}{\hyperlink{class_benchmark}{Benchmark$<$ T $>$} \\*Klasa do przeprowadzenia testów na kontenerach danych }{\pageref{class_benchmark}}{}
\item\contentsline{section}{\hyperlink{class_kolejka}{Kolejka$<$ T\-Y\-P $>$} \\*Klasa \hyperlink{class_kolejka}{Kolejka} }{\pageref{class_kolejka}}{}
\item\contentsline{section}{\hyperlink{class_lista}{Lista$<$ T\-Y\-P $>$} \\*Klasa \hyperlink{class_lista}{Lista} }{\pageref{class_lista}}{}
\item\contentsline{section}{\hyperlink{class_stos}{Stos$<$ T\-Y\-P $>$} \\*Klasa \hyperlink{class_stos}{Stos} }{\pageref{class_stos}}{}
\item\contentsline{section}{\hyperlink{class_timer}{Timer} \\*Klasa do pomiaru różnicy czasów }{\pageref{class_timer}}{}
\end{DoxyCompactList}

\chapter{Indeks plików}
\section{Lista plików}
Tutaj znajduje się lista wszystkich plików z ich krótkimi opisami:\begin{DoxyCompactList}
\item\contentsline{section}{\hyperlink{_benchmark_8cpp}{Benchmark.cpp} (Metody klasy \hyperlink{class_benchmark}{Benchmark} )}{\pageref{_benchmark_8cpp}}{}
\item\contentsline{section}{\hyperlink{_benchmark_8hh}{Benchmark.hh} (Definicja klasy \hyperlink{class_benchmark}{Benchmark} )}{\pageref{_benchmark_8hh}}{}
\item\contentsline{section}{\hyperlink{_kolejka_8hh}{Kolejka.hh} (Definicja klasy \hyperlink{class_kolejka}{Kolejka} )}{\pageref{_kolejka_8hh}}{}
\item\contentsline{section}{\hyperlink{_lista_8cpp}{Lista.cpp} }{\pageref{_lista_8cpp}}{}
\item\contentsline{section}{\hyperlink{_lista_8hh}{Lista.hh} (Definicja klasy \hyperlink{class_lista}{Lista} )}{\pageref{_lista_8hh}}{}
\item\contentsline{section}{\hyperlink{_stos_8hh}{Stos.hh} (Definicja klasy \hyperlink{class_stos}{Stos} )}{\pageref{_stos_8hh}}{}
\item\contentsline{section}{\hyperlink{_struktury_8cpp}{Struktury.cpp} }{\pageref{_struktury_8cpp}}{}
\item\contentsline{section}{\hyperlink{_test_8cpp}{Test.cpp} }{\pageref{_test_8cpp}}{}
\end{DoxyCompactList}

\chapter{Dokumentacja klas}
\hypertarget{class_benchmark}{\section{Dokumentacja szablonu klasy Benchmark$<$ T $>$}
\label{class_benchmark}\index{Benchmark$<$ T $>$@{Benchmark$<$ T $>$}}
}


Klasa do przeprowadzenia testów na kontenerach danych.  




{\ttfamily \#include $<$Benchmark.\-hh$>$}

\subsection*{Metody publiczne}
\begin{DoxyCompactItemize}
\item 
\hyperlink{class_benchmark_a15ccd21b999908da42c58bd5d127c674}{Benchmark} (unsigned int $\ast$test\-\_\-list\-\_\-sizes, unsigned int size, unsigned int number\-\_\-of\-\_\-memeasurements)
\begin{DoxyCompactList}\small\item\em Konstruktor klasy \hyperlink{class_benchmark}{Benchmark}. \end{DoxyCompactList}\item 
void \hyperlink{class_benchmark_ac6a294f9ac4fbba06f0a608eb241d3b0}{test\-Quick\-Sort} (\hyperlink{class_lista}{Lista}$<$ T $>$ $\ast$list, P\-I\-V\-O\-T piv)
\begin{DoxyCompactList}\small\item\em metoda do testowania algorytmu Quick Sort \end{DoxyCompactList}\item 
void \hyperlink{class_benchmark_a90eb1776343e90f4fcbf3fb95aa52bef}{test\-Merge\-Sort} (\hyperlink{class_lista}{Lista}$<$ T $>$ $\ast$list)
\begin{DoxyCompactList}\small\item\em metoda do testowania algorytmyu Merge Sort \end{DoxyCompactList}\end{DoxyCompactItemize}


\subsection{Opis szczegółowy}
\subsubsection*{template$<$class T$>$class Benchmark$<$ T $>$}

Klasa do przeprowadzenia testów na kontenerach danych. 

Klasa odpwiada za pomiar czasu koniecznego do umieszczenia określonej liczby danych na stosie, kolejce oraz liście 

\subsection{Dokumentacja konstruktora i destruktora}
\hypertarget{class_benchmark_a15ccd21b999908da42c58bd5d127c674}{\index{Benchmark@{Benchmark}!Benchmark@{Benchmark}}
\index{Benchmark@{Benchmark}!Benchmark@{Benchmark}}
\subsubsection[{Benchmark}]{\setlength{\rightskip}{0pt plus 5cm}template$<$typename T $>$ {\bf Benchmark}$<$ T $>$\-::{\bf Benchmark} (
\begin{DoxyParamCaption}
\item[{unsigned int $\ast$}]{test\-\_\-list\-\_\-sizes, }
\item[{unsigned int}]{size, }
\item[{unsigned int}]{number\-\_\-of\-\_\-memeasurements}
\end{DoxyParamCaption}
)}}\label{class_benchmark_a15ccd21b999908da42c58bd5d127c674}


Konstruktor klasy \hyperlink{class_benchmark}{Benchmark}. 


\begin{DoxyParams}{Parametry}
{\em test\-\_\-list\-\_\-sizes} & tablica przechowująca rozmiary list do testów \\
\hline
{\em size} & liczba list równych rozmiarów do przetestowania \\
\hline
{\em number\-\_\-of\-\_\-memeasurements} & liczba pomiarów branych do średniej \\
\hline
\end{DoxyParams}


\subsection{Dokumentacja funkcji składowych}
\hypertarget{class_benchmark_a90eb1776343e90f4fcbf3fb95aa52bef}{\index{Benchmark@{Benchmark}!test\-Merge\-Sort@{test\-Merge\-Sort}}
\index{test\-Merge\-Sort@{test\-Merge\-Sort}!Benchmark@{Benchmark}}
\subsubsection[{test\-Merge\-Sort}]{\setlength{\rightskip}{0pt plus 5cm}template$<$typename T $>$ void {\bf Benchmark}$<$ T $>$\-::test\-Merge\-Sort (
\begin{DoxyParamCaption}
\item[{{\bf Lista}$<$ T $>$ $\ast$}]{list}
\end{DoxyParamCaption}
)}}\label{class_benchmark_a90eb1776343e90f4fcbf3fb95aa52bef}


metoda do testowania algorytmyu Merge Sort 


\begin{DoxyParams}{Parametry}
{\em list} & lista na której testujemy sortowanie \\
\hline
\end{DoxyParams}
\hypertarget{class_benchmark_ac6a294f9ac4fbba06f0a608eb241d3b0}{\index{Benchmark@{Benchmark}!test\-Quick\-Sort@{test\-Quick\-Sort}}
\index{test\-Quick\-Sort@{test\-Quick\-Sort}!Benchmark@{Benchmark}}
\subsubsection[{test\-Quick\-Sort}]{\setlength{\rightskip}{0pt plus 5cm}template$<$typename T $>$ void {\bf Benchmark}$<$ T $>$\-::test\-Quick\-Sort (
\begin{DoxyParamCaption}
\item[{{\bf Lista}$<$ T $>$ $\ast$}]{list, }
\item[{P\-I\-V\-O\-T}]{piv}
\end{DoxyParamCaption}
)}}\label{class_benchmark_ac6a294f9ac4fbba06f0a608eb241d3b0}


metoda do testowania algorytmu Quick Sort 


\begin{DoxyParams}{Parametry}
{\em list} & lista na której testujemy sortowanie \\
\hline
{\em piv} & jaką motedę wyboru elementu osiowego wybieramy \\
\hline
\end{DoxyParams}


Dokumentacja dla tej klasy została wygenerowana z pliku\-:\begin{DoxyCompactItemize}
\item 
/home/mateusz/git/209365/\-Lab3/prj/inc/\hyperlink{_benchmark_8hh}{Benchmark.\-hh}\end{DoxyCompactItemize}

\hypertarget{class_element}{
\section{Dokumentacja szablonu klasy Element$<$ TYP $>$}
\label{class_element}\index{Element@{Element}}
}


Klasa \hyperlink{class_element}{Element}.  




{\ttfamily \#include $<$Element.hh$>$}



Diagram współpracy dla Element$<$ TYP $>$:
\nopagebreak
\begin{figure}[H]
\begin{center}
\leavevmode
\includegraphics[width=228pt]{class_element__coll__graph}
\end{center}
\end{figure}
\subsection*{Metody publiczne}
\begin{DoxyCompactItemize}
\item 
\hyperlink{class_element_a92c7c88e8ee87743453ef0af189d169e}{Element} ()
\item 
\hyperlink{class_element_a10ef9400ca8edfe61fe67149d1994a14}{$\sim$Element} ()
\end{DoxyCompactItemize}
\subsection*{Atrybuty publiczne}
\begin{DoxyCompactItemize}
\item 
TYP \hyperlink{class_element_a686c632e35d9fe961d0bf5482e1215fd}{wartosc}
\item 
\hyperlink{class_element}{Element} $\ast$ \hyperlink{class_element_a1af602058c187cccf3f50d0c1a21047d}{nastepny}
\end{DoxyCompactItemize}


\subsection{Opis szczegółowy}
\subsubsection*{template$<$class TYP$>$ class Element$<$ TYP $>$}

Klasa ta modeluje nam pojedyncza dana oraz wskaznik na kolejna dana 
\begin{DoxyParams}{Parametry}
\item[\mbox{\tt[in]} {\em $\ast$nastepny}]-\/ jako wskaznik na kolejny element \item[\mbox{\tt[in]} {\em wartosc}]-\/ wartosc naszego elementu zdefiniowana przez nas przy uzyciu szablonu \end{DoxyParams}


Definicja w linii 22 pliku Element.hh.



\subsection{Dokumentacja konstruktora i destruktora}
\hypertarget{class_element_a92c7c88e8ee87743453ef0af189d169e}{
\index{Element@{Element}!Element@{Element}}
\index{Element@{Element}!Element@{Element}}
\subsubsection[{Element}]{\setlength{\rightskip}{0pt plus 5cm}template$<$class TYP$>$ {\bf Element}$<$ TYP $>$::{\bf Element} (
\begin{DoxyParamCaption}
{}
\end{DoxyParamCaption}
)\hspace{0.3cm}{\ttfamily  \mbox{[}inline\mbox{]}}}}
\label{class_element_a92c7c88e8ee87743453ef0af189d169e}


Definicja w linii 29 pliku Element.hh.

\hypertarget{class_element_a10ef9400ca8edfe61fe67149d1994a14}{
\index{Element@{Element}!$\sim$Element@{$\sim$Element}}
\index{$\sim$Element@{$\sim$Element}!Element@{Element}}
\subsubsection[{$\sim$Element}]{\setlength{\rightskip}{0pt plus 5cm}template$<$class TYP$>$ {\bf Element}$<$ TYP $>$::$\sim${\bf Element} (
\begin{DoxyParamCaption}
{}
\end{DoxyParamCaption}
)\hspace{0.3cm}{\ttfamily  \mbox{[}inline\mbox{]}}}}
\label{class_element_a10ef9400ca8edfe61fe67149d1994a14}


Definicja w linii 34 pliku Element.hh.



\subsection{Dokumentacja atrybutów składowych}
\hypertarget{class_element_a1af602058c187cccf3f50d0c1a21047d}{
\index{Element@{Element}!nastepny@{nastepny}}
\index{nastepny@{nastepny}!Element@{Element}}
\subsubsection[{nastepny}]{\setlength{\rightskip}{0pt plus 5cm}template$<$class TYP$>$ {\bf Element}$\ast$ {\bf Element}$<$ TYP $>$::{\bf nastepny}}}
\label{class_element_a1af602058c187cccf3f50d0c1a21047d}


Definicja w linii 27 pliku Element.hh.

\hypertarget{class_element_a686c632e35d9fe961d0bf5482e1215fd}{
\index{Element@{Element}!wartosc@{wartosc}}
\index{wartosc@{wartosc}!Element@{Element}}
\subsubsection[{wartosc}]{\setlength{\rightskip}{0pt plus 5cm}template$<$class TYP$>$ TYP {\bf Element}$<$ TYP $>$::{\bf wartosc}}}
\label{class_element_a686c632e35d9fe961d0bf5482e1215fd}


Definicja w linii 26 pliku Element.hh.



Dokumentacja dla tej klasy została wygenerowana z pliku:\begin{DoxyCompactItemize}
\item 
\hyperlink{_element_8hh}{Element.hh}\end{DoxyCompactItemize}

\hypertarget{class_kolejka}{\section{Dokumentacja szablonu klasy Kolejka$<$ T\-Y\-P $>$}
\label{class_kolejka}\index{Kolejka$<$ T\-Y\-P $>$@{Kolejka$<$ T\-Y\-P $>$}}
}


Klasa \hyperlink{class_kolejka}{Kolejka}.  




{\ttfamily \#include $<$Kolejka.\-hh$>$}

Diagram dziedziczenia dla Kolejka$<$ T\-Y\-P $>$\begin{figure}[H]
\begin{center}
\leavevmode
\includegraphics[height=2.000000cm]{class_kolejka}
\end{center}
\end{figure}
\subsection*{Metody publiczne}
\begin{DoxyCompactItemize}
\item 
\hypertarget{class_kolejka_a66056e67a0514466d7771da351e8468b}{void {\bfseries P\-U\-S\-H} (T\-Y\-P liczba)}\label{class_kolejka_a66056e67a0514466d7771da351e8468b}

\item 
\hypertarget{class_kolejka_ab1d8e5d4a855fb2156201a57cc0a9a39}{int {\bfseries P\-O\-P} ()}\label{class_kolejka_ab1d8e5d4a855fb2156201a57cc0a9a39}

\item 
\hypertarget{class_kolejka_a26b5afbcc9f892a41acbd3aed062ee50}{void {\bfseries S\-H\-O\-W} ()}\label{class_kolejka_a26b5afbcc9f892a41acbd3aed062ee50}

\item 
\hypertarget{class_kolejka_a06a7fe157ff434771a700ffd084dc7a2}{unsigned int {\bfseries S\-I\-Z\-E} ()}\label{class_kolejka_a06a7fe157ff434771a700ffd084dc7a2}

\end{DoxyCompactItemize}


\subsection{Opis szczegółowy}
\subsubsection*{template$<$typename T\-Y\-P$>$class Kolejka$<$ T\-Y\-P $>$}

Klasa \hyperlink{class_kolejka}{Kolejka}. 

Klasa ta modeluje nam Kolejke Składa się z pól klasy \hyperlink{class_lista}{Lista} oraz metod P\-U\-S\-H, P\-O\-P, S\-I\-Z\-E, S\-H\-O\-W Klasa w calosci wykorzystuje implementacje listy 

Dokumentacja dla tej klasy została wygenerowana z pliku\-:\begin{DoxyCompactItemize}
\item 
/home/mateusz/git/209365/\-Lab3/prj/inc/\hyperlink{_kolejka_8hh}{Kolejka.\-hh}\end{DoxyCompactItemize}

\hypertarget{class_lista}{
\section{Dokumentacja szablonu klasy Lista$<$ TYP $>$}
\label{class_lista}\index{Lista@{Lista}}
}


Klasa \hyperlink{class_lista}{Lista}.  




{\ttfamily \#include $<$Lista.hh$>$}



Diagram dziedziczenia dla Lista$<$ TYP $>$
\nopagebreak
\begin{figure}[H]
\begin{center}
\leavevmode
\includegraphics[width=258pt]{class_lista__inherit__graph}
\end{center}
\end{figure}
\subsection*{Metody publiczne}
\begin{DoxyCompactItemize}
\item 
\hyperlink{class_lista_ae2559bf0c96569265d5f9f7f9cd4cb3a}{Lista} ()
\item 
\hyperlink{class_lista_a7271d867e38a838c25ffb5aa6b73faef}{$\sim$Lista} ()
\item 
void \hyperlink{class_lista_a5c21bab22b627c729ad0c90e1f835901}{PUSH} (TYP liczba)
\item 
int \hyperlink{class_lista_a99390f363a7deceed9c99d717f578aaa}{POP} (unsigned int Numer\_\-Elementu)
\item 
unsigned int \hyperlink{class_lista_a4f10ca015c6b34a322dbc1c93e313c07}{SIZE} ()
\item 
void \hyperlink{class_lista_a89bbb449a047593eebce602a449ac1e7}{SHOW} ()
\end{DoxyCompactItemize}
\subsection*{Atrybuty chronione}
\begin{DoxyCompactItemize}
\item 
\hyperlink{class_element}{Element}$<$ TYP $>$ $\ast$ \hyperlink{class_lista_a8557ec530a4dcf731449d708426fe5d5}{pierwszy}
\item 
\hyperlink{class_element}{Element}$<$ TYP $>$ $\ast$ \hyperlink{class_lista_aefe3a0b196bd94713eebe3f9332011b1}{ostatni}
\end{DoxyCompactItemize}


\subsection{Opis szczegółowy}
\subsubsection*{template$<$typename TYP$>$ class Lista$<$ TYP $>$}

Klasa ta modeluje nam Liste Składa się z metod PUSH, POP, SIZE, SHOW oraz pól: 
\begin{DoxyParams}{Parametry}
\item[\mbox{\tt[in]} {\em $\ast$pierwszy}]-\/ jako wskaznik na pierwszy element \end{DoxyParams}


Definicja w linii 27 pliku Lista.hh.



\subsection{Dokumentacja konstruktora i destruktora}
\hypertarget{class_lista_ae2559bf0c96569265d5f9f7f9cd4cb3a}{
\index{Lista@{Lista}!Lista@{Lista}}
\index{Lista@{Lista}!Lista@{Lista}}
\subsubsection[{Lista}]{\setlength{\rightskip}{0pt plus 5cm}template$<$typename TYP$>$ {\bf Lista}$<$ TYP $>$::{\bf Lista} (
\begin{DoxyParamCaption}
{}
\end{DoxyParamCaption}
)\hspace{0.3cm}{\ttfamily  \mbox{[}inline\mbox{]}}}}
\label{class_lista_ae2559bf0c96569265d5f9f7f9cd4cb3a}


Definicja w linii 33 pliku Lista.hh.

\hypertarget{class_lista_a7271d867e38a838c25ffb5aa6b73faef}{
\index{Lista@{Lista}!$\sim$Lista@{$\sim$Lista}}
\index{$\sim$Lista@{$\sim$Lista}!Lista@{Lista}}
\subsubsection[{$\sim$Lista}]{\setlength{\rightskip}{0pt plus 5cm}template$<$typename TYP$>$ {\bf Lista}$<$ TYP $>$::$\sim${\bf Lista} (
\begin{DoxyParamCaption}
{}
\end{DoxyParamCaption}
)\hspace{0.3cm}{\ttfamily  \mbox{[}inline\mbox{]}}}}
\label{class_lista_a7271d867e38a838c25ffb5aa6b73faef}


Definicja w linii 34 pliku Lista.hh.



\subsection{Dokumentacja funkcji składowych}
\hypertarget{class_lista_a99390f363a7deceed9c99d717f578aaa}{
\index{Lista@{Lista}!POP@{POP}}
\index{POP@{POP}!Lista@{Lista}}
\subsubsection[{POP}]{\setlength{\rightskip}{0pt plus 5cm}template$<$typename TYP$>$ int {\bf Lista}$<$ TYP $>$::POP (
\begin{DoxyParamCaption}
\item[{unsigned int}]{ Numer\_\-Elementu}
\end{DoxyParamCaption}
)\hspace{0.3cm}{\ttfamily  \mbox{[}inline\mbox{]}}}}
\label{class_lista_a99390f363a7deceed9c99d717f578aaa}


Definicja w linii 59 pliku Lista.hh.



Oto graf wywoływań tej funkcji:
\nopagebreak
\begin{figure}[H]
\begin{center}
\leavevmode
\includegraphics[width=220pt]{class_lista_a99390f363a7deceed9c99d717f578aaa_icgraph}
\end{center}
\end{figure}


\hypertarget{class_lista_a5c21bab22b627c729ad0c90e1f835901}{
\index{Lista@{Lista}!PUSH@{PUSH}}
\index{PUSH@{PUSH}!Lista@{Lista}}
\subsubsection[{PUSH}]{\setlength{\rightskip}{0pt plus 5cm}template$<$typename TYP$>$ void {\bf Lista}$<$ TYP $>$::PUSH (
\begin{DoxyParamCaption}
\item[{TYP}]{ liczba}
\end{DoxyParamCaption}
)\hspace{0.3cm}{\ttfamily  \mbox{[}inline\mbox{]}}}}
\label{class_lista_a5c21bab22b627c729ad0c90e1f835901}


Reimplementowana w \hyperlink{class_kolejka_a66056e67a0514466d7771da351e8468b}{Kolejka$<$ TYP $>$} i \hyperlink{class_stos_a773cf22cb5c67bda0d2878a1ec8bc363}{Stos$<$ TYP $>$}.



Definicja w linii 44 pliku Lista.hh.



Oto graf wywoływań tej funkcji:
\nopagebreak
\begin{figure}[H]
\begin{center}
\leavevmode
\includegraphics[width=226pt]{class_lista_a5c21bab22b627c729ad0c90e1f835901_icgraph}
\end{center}
\end{figure}


\hypertarget{class_lista_a89bbb449a047593eebce602a449ac1e7}{
\index{Lista@{Lista}!SHOW@{SHOW}}
\index{SHOW@{SHOW}!Lista@{Lista}}
\subsubsection[{SHOW}]{\setlength{\rightskip}{0pt plus 5cm}template$<$typename TYP$>$ void {\bf Lista}$<$ TYP $>$::SHOW (
\begin{DoxyParamCaption}
{}
\end{DoxyParamCaption}
)\hspace{0.3cm}{\ttfamily  \mbox{[}inline\mbox{]}}}}
\label{class_lista_a89bbb449a047593eebce602a449ac1e7}


Reimplementowana w \hyperlink{class_kolejka_a26b5afbcc9f892a41acbd3aed062ee50}{Kolejka$<$ TYP $>$} i \hyperlink{class_stos_a8f1c40b779a699c84b20eeb59ca67f06}{Stos$<$ TYP $>$}.



Definicja w linii 108 pliku Lista.hh.

\hypertarget{class_lista_a4f10ca015c6b34a322dbc1c93e313c07}{
\index{Lista@{Lista}!SIZE@{SIZE}}
\index{SIZE@{SIZE}!Lista@{Lista}}
\subsubsection[{SIZE}]{\setlength{\rightskip}{0pt plus 5cm}template$<$typename TYP$>$ unsigned int {\bf Lista}$<$ TYP $>$::SIZE (
\begin{DoxyParamCaption}
{}
\end{DoxyParamCaption}
)\hspace{0.3cm}{\ttfamily  \mbox{[}inline\mbox{]}}}}
\label{class_lista_a4f10ca015c6b34a322dbc1c93e313c07}


Reimplementowana w \hyperlink{class_kolejka_a06a7fe157ff434771a700ffd084dc7a2}{Kolejka$<$ TYP $>$} i \hyperlink{class_stos_a6ff0d2aa5946c0dc413e3236ca99fd26}{Stos$<$ TYP $>$}.



Definicja w linii 88 pliku Lista.hh.



\subsection{Dokumentacja atrybutów składowych}
\hypertarget{class_lista_aefe3a0b196bd94713eebe3f9332011b1}{
\index{Lista@{Lista}!ostatni@{ostatni}}
\index{ostatni@{ostatni}!Lista@{Lista}}
\subsubsection[{ostatni}]{\setlength{\rightskip}{0pt plus 5cm}template$<$typename TYP$>$ {\bf Element}$<$TYP$>$$\ast$ {\bf Lista}$<$ TYP $>$::{\bf ostatni}\hspace{0.3cm}{\ttfamily  \mbox{[}protected\mbox{]}}}}
\label{class_lista_aefe3a0b196bd94713eebe3f9332011b1}


Definicja w linii 31 pliku Lista.hh.

\hypertarget{class_lista_a8557ec530a4dcf731449d708426fe5d5}{
\index{Lista@{Lista}!pierwszy@{pierwszy}}
\index{pierwszy@{pierwszy}!Lista@{Lista}}
\subsubsection[{pierwszy}]{\setlength{\rightskip}{0pt plus 5cm}template$<$typename TYP$>$ {\bf Element}$<$TYP$>$$\ast$ {\bf Lista}$<$ TYP $>$::{\bf pierwszy}\hspace{0.3cm}{\ttfamily  \mbox{[}protected\mbox{]}}}}
\label{class_lista_a8557ec530a4dcf731449d708426fe5d5}


Definicja w linii 30 pliku Lista.hh.



Dokumentacja dla tej klasy została wygenerowana z pliku:\begin{DoxyCompactItemize}
\item 
\hyperlink{_lista_8hh}{Lista.hh}\end{DoxyCompactItemize}

\hypertarget{class_stos}{\section{Dokumentacja szablonu klasy Stos$<$ T\-Y\-P $>$}
\label{class_stos}\index{Stos$<$ T\-Y\-P $>$@{Stos$<$ T\-Y\-P $>$}}
}


Klasa \hyperlink{class_stos}{Stos}.  




{\ttfamily \#include $<$Stos.\-hh$>$}

Diagram dziedziczenia dla Stos$<$ T\-Y\-P $>$\begin{figure}[H]
\begin{center}
\leavevmode
\includegraphics[height=2.000000cm]{class_stos}
\end{center}
\end{figure}
\subsection*{Metody publiczne}
\begin{DoxyCompactItemize}
\item 
\hypertarget{class_stos_a773cf22cb5c67bda0d2878a1ec8bc363}{void {\bfseries P\-U\-S\-H} (T\-Y\-P liczba)}\label{class_stos_a773cf22cb5c67bda0d2878a1ec8bc363}

\item 
\hypertarget{class_stos_ab8b0ecec7cbe0ae761bfee9d9e47d5d4}{int {\bfseries P\-O\-P} ()}\label{class_stos_ab8b0ecec7cbe0ae761bfee9d9e47d5d4}

\item 
\hypertarget{class_stos_a8f1c40b779a699c84b20eeb59ca67f06}{void {\bfseries S\-H\-O\-W} ()}\label{class_stos_a8f1c40b779a699c84b20eeb59ca67f06}

\item 
\hypertarget{class_stos_a6ff0d2aa5946c0dc413e3236ca99fd26}{unsigned int {\bfseries S\-I\-Z\-E} ()}\label{class_stos_a6ff0d2aa5946c0dc413e3236ca99fd26}

\end{DoxyCompactItemize}
\subsection*{Dodatkowe Dziedziczone Składowe}


\subsection{Opis szczegółowy}
\subsubsection*{template$<$typename T\-Y\-P$>$class Stos$<$ T\-Y\-P $>$}

Klasa \hyperlink{class_stos}{Stos}. 

Klasa ta modeluje nam \hyperlink{class_stos}{Stos} Składa się z pól klasy \hyperlink{class_lista}{Lista} ktore zostania uzyte oraz metod P\-U\-S\-H, P\-O\-P, S\-I\-Z\-E, S\-H\-O\-W Klasa w calosci wykorzystuje implementacje listy 

Dokumentacja dla tej klasy została wygenerowana z pliku\-:\begin{DoxyCompactItemize}
\item 
/home/mateusz/git/209365/\-Lab3/prj/inc/\hyperlink{_stos_8hh}{Stos.\-hh}\end{DoxyCompactItemize}

\chapter{Dokumentacja plików}
\hypertarget{_benchmark_8cpp}{
\section{Dokumentacja pliku Benchmark.cpp}
\label{_benchmark_8cpp}\index{Benchmark.cpp@{Benchmark.cpp}}
}


Metody klasy \hyperlink{class_benchmark}{Benchmark}.  


{\ttfamily \#include \char`\"{}Benchmark.hh\char`\"{}}\par
{\ttfamily \#include $<$cstdlib$>$}\par
{\ttfamily \#include $<$fstream$>$}\par
{\ttfamily \#include $<$ctime$>$}\par
Wykres zależności załączania dla Benchmark.cpp:
\nopagebreak
\begin{figure}[H]
\begin{center}
\leavevmode
\includegraphics[width=348pt]{_benchmark_8cpp__incl}
\end{center}
\end{figure}


\subsection{Opis szczegółowy}
Plik zawiera definicje metod klasy \hyperlink{class_benchmark}{Benchmark} 

Definicja w pliku \hyperlink{_benchmark_8cpp_source}{Benchmark.cpp}.


\hypertarget{_benchmark_8hh}{\section{Dokumentacja pliku /home/mateusz/git/209365/\-Lab3/prj/inc/\-Benchmark.hh}
\label{_benchmark_8hh}\index{/home/mateusz/git/209365/\-Lab3/prj/inc/\-Benchmark.\-hh@{/home/mateusz/git/209365/\-Lab3/prj/inc/\-Benchmark.\-hh}}
}


Deklaracja i definicja (razem, bo szablon) klasy \hyperlink{class_benchmark}{Benchmark}.  


{\ttfamily \#include \char`\"{}../inc/\-Lista.\-hh\char`\"{}}\\*
{\ttfamily \#include \char`\"{}../inc/quick\-\_\-sort.\-hh\char`\"{}}\\*
{\ttfamily \#include \char`\"{}../inc/merge\-\_\-sort.\-hh\char`\"{}}\\*
{\ttfamily \#include \char`\"{}../inc/\-Timer.\-hh\char`\"{}}\\*
{\ttfamily \#include $<$iostream$>$}\\*
{\ttfamily \#include $<$cstdlib$>$}\\*
{\ttfamily \#include $<$cstdio$>$}\\*
{\ttfamily \#include $<$ctime$>$}\\*
{\ttfamily \#include $<$string$>$}\\*
\subsection*{Komponenty}
\begin{DoxyCompactItemize}
\item 
class \hyperlink{class_benchmark}{Benchmark$<$ T $>$}
\begin{DoxyCompactList}\small\item\em Klasa do przeprowadzenia testów na kontenerach danych. \end{DoxyCompactList}\end{DoxyCompactItemize}


\subsection{Opis szczegółowy}
Deklaracja i definicja (razem, bo szablon) klasy \hyperlink{class_benchmark}{Benchmark}. \hyperlink{_benchmark_8hh}{Benchmark.\-hh} 
\hypertarget{_element_8hh}{
\section{Dokumentacja pliku Element.hh}
\label{_element_8hh}\index{Element.hh@{Element.hh}}
}


Definicja klasy \hyperlink{class_element}{Element}.  


Ten wykres pokazuje, które pliki bezpośrednio lub pośrednio załączają ten plik:
\nopagebreak
\begin{figure}[H]
\begin{center}
\leavevmode
\includegraphics[width=286pt]{_element_8hh__dep__incl}
\end{center}
\end{figure}
\subsection*{Komponenty}
\begin{DoxyCompactItemize}
\item 
class \hyperlink{class_element}{Element$<$ TYP $>$}
\begin{DoxyCompactList}\small\item\em Klasa \hyperlink{class_element}{Element}. \item\end{DoxyCompactList}\end{DoxyCompactItemize}


\subsection{Opis szczegółowy}
Plik zawiera definicje klasy \hyperlink{class_element}{Element}, ktora bedzie pojedynczym elementem naszej struktury. Klasa ta posiada szablon, dzieki czemu mozemy pracowac na roznych typach danych 

Definicja w pliku \hyperlink{_element_8hh_source}{Element.hh}.


\hypertarget{_kolejka_8hh}{\section{Dokumentacja pliku /home/mateusz/git/209365/\-Lab3/prj/inc/\-Kolejka.hh}
\label{_kolejka_8hh}\index{/home/mateusz/git/209365/\-Lab3/prj/inc/\-Kolejka.\-hh@{/home/mateusz/git/209365/\-Lab3/prj/inc/\-Kolejka.\-hh}}
}


Definicja klasy \hyperlink{class_kolejka}{Kolejka}.  


{\ttfamily \#include $<$iostream$>$}\\*
{\ttfamily \#include \char`\"{}Lista.\-hh\char`\"{}}\\*
\subsection*{Komponenty}
\begin{DoxyCompactItemize}
\item 
class \hyperlink{class_kolejka}{Kolejka$<$ T\-Y\-P $>$}
\begin{DoxyCompactList}\small\item\em Klasa \hyperlink{class_kolejka}{Kolejka}. \end{DoxyCompactList}\end{DoxyCompactItemize}


\subsection{Opis szczegółowy}
Definicja klasy \hyperlink{class_kolejka}{Kolejka}. Plik zawiera definicje klasy \hyperlink{class_kolejka}{Kolejka}, ktora bedzie struktura naszych danych. Klasa ta posiada szablon, dzieki czemu mozemy pracowac na roznych typach danych 
\hypertarget{_lista_8hh}{
\section{Dokumentacja pliku Lista.hh}
\label{_lista_8hh}\index{Lista.hh@{Lista.hh}}
}


Definicja klasy \hyperlink{class_lista}{Lista}.  


{\ttfamily \#include $<$iostream$>$}\par
{\ttfamily \#include \char`\"{}Element.hh\char`\"{}}\par
Wykres zależności załączania dla Lista.hh:
\nopagebreak
\begin{figure}[H]
\begin{center}
\leavevmode
\includegraphics[width=218pt]{_lista_8hh__incl}
\end{center}
\end{figure}
Ten wykres pokazuje, które pliki bezpośrednio lub pośrednio załączają ten plik:
\nopagebreak
\begin{figure}[H]
\begin{center}
\leavevmode
\includegraphics[width=235pt]{_lista_8hh__dep__incl}
\end{center}
\end{figure}
\subsection*{Komponenty}
\begin{DoxyCompactItemize}
\item 
class \hyperlink{class_lista}{Lista$<$ TYP $>$}
\begin{DoxyCompactList}\small\item\em Klasa \hyperlink{class_lista}{Lista}. \item\end{DoxyCompactList}\end{DoxyCompactItemize}


\subsection{Opis szczegółowy}
Plik zawiera definicje klasy \hyperlink{class_lista}{Lista}, ktora bedzie struktura naszych danych. Klasa ta posiada szablon, dzieki czemu mozemy pracowac na roznych typach danych 

Definicja w pliku \hyperlink{_lista_8hh_source}{Lista.hh}.


\hypertarget{_stos_8hh}{
\section{Dokumentacja pliku Stos.hh}
\label{_stos_8hh}\index{Stos.hh@{Stos.hh}}
}


Definicja klasy \hyperlink{class_stos}{Stos}.  


{\ttfamily \#include $<$iostream$>$}\par
{\ttfamily \#include \char`\"{}Element.hh\char`\"{}}\par
{\ttfamily \#include \char`\"{}Lista.hh\char`\"{}}\par
Wykres zależności załączania dla Stos.hh:
\nopagebreak
\begin{figure}[H]
\begin{center}
\leavevmode
\includegraphics[width=232pt]{_stos_8hh__incl}
\end{center}
\end{figure}
Ten wykres pokazuje, które pliki bezpośrednio lub pośrednio załączają ten plik:
\nopagebreak
\begin{figure}[H]
\begin{center}
\leavevmode
\includegraphics[width=226pt]{_stos_8hh__dep__incl}
\end{center}
\end{figure}
\subsection*{Komponenty}
\begin{DoxyCompactItemize}
\item 
class \hyperlink{class_stos}{Stos$<$ TYP $>$}
\begin{DoxyCompactList}\small\item\em Klasa \hyperlink{class_stos}{Stos}. \item\end{DoxyCompactList}\end{DoxyCompactItemize}


\subsection{Opis szczegółowy}
Plik zawiera definicje klasy \hyperlink{class_stos}{Stos}, ktora bedzie struktura naszych danych. Klasa ta posiada szablon, dzieki czemu mozemy pracowac na roznych typach danych 

Definicja w pliku \hyperlink{_stos_8hh_source}{Stos.hh}.


\input{strona_8dox}
\hypertarget{_struktury_8cpp}{
\section{Dokumentacja pliku Struktury.cpp}
\label{_struktury_8cpp}\index{Struktury.cpp@{Struktury.cpp}}
}
{\ttfamily \#include $<$iostream$>$}\par
{\ttfamily \#include \char`\"{}Lista.hh\char`\"{}}\par
{\ttfamily \#include \char`\"{}Kolejka.hh\char`\"{}}\par
{\ttfamily \#include \char`\"{}Stos.hh\char`\"{}}\par
Wykres zależności załączania dla Struktury.cpp:
\nopagebreak
\begin{figure}[H]
\begin{center}
\leavevmode
\includegraphics[width=278pt]{_struktury_8cpp__incl}
\end{center}
\end{figure}
\subsection*{Funkcje}
\begin{DoxyCompactItemize}
\item 
int \hyperlink{_struktury_8cpp_ae66f6b31b5ad750f1fe042a706a4e3d4}{main} ()
\end{DoxyCompactItemize}


\subsection{Dokumentacja funkcji}
\hypertarget{_struktury_8cpp_ae66f6b31b5ad750f1fe042a706a4e3d4}{
\index{Struktury.cpp@{Struktury.cpp}!main@{main}}
\index{main@{main}!Struktury.cpp@{Struktury.cpp}}
\subsubsection[{main}]{\setlength{\rightskip}{0pt plus 5cm}int main (
\begin{DoxyParamCaption}
{}
\end{DoxyParamCaption}
)}}
\label{_struktury_8cpp_ae66f6b31b5ad750f1fe042a706a4e3d4}


Definicja w linii 8 pliku Struktury.cpp.



Oto graf wywołań dla tej funkcji:
\nopagebreak
\begin{figure}[H]
\begin{center}
\leavevmode
\includegraphics[width=228pt]{_struktury_8cpp_ae66f6b31b5ad750f1fe042a706a4e3d4_cgraph}
\end{center}
\end{figure}



\hypertarget{_test_8cpp}{
\section{Dokumentacja pliku Test.cpp}
\label{_test_8cpp}\index{Test.cpp@{Test.cpp}}
}
{\ttfamily \#include $<$iostream$>$}\par
{\ttfamily \#include \char`\"{}Benchmark.hh\char`\"{}}\par
{\ttfamily \#include \char`\"{}Lista.hh\char`\"{}}\par
{\ttfamily \#include \char`\"{}Kolejka.hh\char`\"{}}\par
{\ttfamily \#include \char`\"{}Stos.hh\char`\"{}}\par
Wykres zależności załączania dla Test.cpp:
\nopagebreak
\begin{figure}[H]
\begin{center}
\leavevmode
\includegraphics[width=376pt]{_test_8cpp__incl}
\end{center}
\end{figure}
\subsection*{Definicje}
\begin{DoxyCompactItemize}
\item 
\#define \hyperlink{_test_8cpp_a77b984cfb88c2ca29fea115a7019ba8f}{STALA}~10
\end{DoxyCompactItemize}
\subsection*{Funkcje}
\begin{DoxyCompactItemize}
\item 
double \hyperlink{_test_8cpp_ae36bd4cd4ba4a0663e79b579bf31fc20}{funkcja} (double x)
\item 
int \hyperlink{_test_8cpp_ae66f6b31b5ad750f1fe042a706a4e3d4}{main} ()
\end{DoxyCompactItemize}


\subsection{Dokumentacja definicji}
\hypertarget{_test_8cpp_a77b984cfb88c2ca29fea115a7019ba8f}{
\index{Test.cpp@{Test.cpp}!STALA@{STALA}}
\index{STALA@{STALA}!Test.cpp@{Test.cpp}}
\subsubsection[{STALA}]{\setlength{\rightskip}{0pt plus 5cm}\#define STALA~10}}
\label{_test_8cpp_a77b984cfb88c2ca29fea115a7019ba8f}


Definicja w linii 7 pliku Test.cpp.



\subsection{Dokumentacja funkcji}
\hypertarget{_test_8cpp_ae36bd4cd4ba4a0663e79b579bf31fc20}{
\index{Test.cpp@{Test.cpp}!funkcja@{funkcja}}
\index{funkcja@{funkcja}!Test.cpp@{Test.cpp}}
\subsubsection[{funkcja}]{\setlength{\rightskip}{0pt plus 5cm}double funkcja (
\begin{DoxyParamCaption}
\item[{double}]{ x}
\end{DoxyParamCaption}
)}}
\label{_test_8cpp_ae36bd4cd4ba4a0663e79b579bf31fc20}


Definicja w linii 10 pliku Test.cpp.

\hypertarget{_test_8cpp_ae66f6b31b5ad750f1fe042a706a4e3d4}{
\index{Test.cpp@{Test.cpp}!main@{main}}
\index{main@{main}!Test.cpp@{Test.cpp}}
\subsubsection[{main}]{\setlength{\rightskip}{0pt plus 5cm}int main (
\begin{DoxyParamCaption}
{}
\end{DoxyParamCaption}
)}}
\label{_test_8cpp_ae66f6b31b5ad750f1fe042a706a4e3d4}


Definicja w linii 14 pliku Test.cpp.



Oto graf wywołań dla tej funkcji:
\nopagebreak
\begin{figure}[H]
\begin{center}
\leavevmode
\includegraphics[width=292pt]{_test_8cpp_ae66f6b31b5ad750f1fe042a706a4e3d4_cgraph}
\end{center}
\end{figure}



\printindex
\end{document}
